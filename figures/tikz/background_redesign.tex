\documentclass[tikz,border=10pt]{standalone}
\usepackage{tikz}
\usepackage{xcolor}
\usepackage{amsmath}
\usepackage{amssymb}
\usetikzlibrary{arrows.meta}
\usetikzlibrary{shapes.geometric, arrows.meta, positioning, shadows, patterns, decorations.pathreplacing, calc, fadings, matrix}

% Define professional color palette
\definecolor{natblue}{RGB}{41,128,185}
\definecolor{natgreen}{RGB}{39,174,96}
\definecolor{natorange}{RGB}{230,126,34}
\definecolor{natpurple}{RGB}{142,68,173}
\definecolor{natred}{RGB}{231,76,60}
\definecolor{natgray}{RGB}{52,73,94}
\definecolor{lightgray}{RGB}{236,240,241}
\definecolor{mdacolor}{RGB}{255,87,87}  % MDA specific color
\definecolor{benefitcolor}{RGB}{46,204,113} % Benefits
\definecolor{challengecolor}{RGB}{231,76,60} % Challenges

\begin{document}

% Define styles
\tikzset{
	% Main process blocks
	processblock/.style={
			rectangle,
			rounded corners=15pt,
			draw=#1,
			line width=2pt,
			top color=#1!20,
			bottom color=#1!5,
			minimum height=3.5cm,
			minimum width=4cm,
			text centered,
			font=\sffamily\large\bfseries,
			drop shadow={shadow scale=1.08, shadow xshift=3pt, shadow yshift=-3pt, fill=#1!40}
		},
	% Feature boxes
	featurebox/.style={
			rectangle,
			rounded corners=10pt,
			draw=#1,
			line width=1.5pt,
			fill=#1!15,
			minimum height=2.5cm,
			minimum width=3.2cm,
			text centered,
			font=\sffamily\normalsize,
			drop shadow={shadow scale=1.05, shadow xshift=2pt, shadow yshift=-2pt, fill=#1!25}
		},
	% Solution box
	solutionbox/.style={
			rectangle,
			rounded corners=15pt,
			draw=natpurple,
			line width=3pt,
			top color=natpurple!20,
			bottom color=natpurple!5,
			minimum height=3cm,
			minimum width=4cm,
			text centered,
			font=\sffamily\Large\bfseries,
			drop shadow={shadow scale=1.1, shadow xshift=4pt, shadow yshift=-4pt, fill=natpurple!40}
		},
	% Analysis box
	analysisbox/.style={
			rectangle,
			rounded corners=12pt,
			draw=#1,
			line width=2pt,
			top color=#1!25,
			bottom color=#1!8,
			minimum height=3cm,
			minimum width=5cm,
			text centered,
			font=\sffamily\large\bfseries,
			drop shadow={shadow scale=1.06, shadow xshift=2pt, shadow yshift=-2pt, fill=#1!30}
		},
	% Arrow style
	arrow/.style={
			->,
			>=stealth,
			line width=2.5pt,
			#1,
			shorten >=3pt,
			shorten <=3pt
		},
	% Title style
	titlebox/.style={
			rectangle,
			rounded corners=12pt,
			draw=natblue,
			line width=2.5pt,
			top color=natblue!15,
			bottom color=white,
			font=\sffamily\Huge\bfseries,
			text=natblue!90,
			inner sep=12pt,
			drop shadow={shadow scale=1.05, shadow xshift=3pt, shadow yshift=-3pt, fill=natblue!30}
		}
}

\begin{tikzpicture}[node distance=2.5cm, every node/.style={transform shape}]

	% Title
	\node[titlebox] (title) {ChimeraLM: Background \& Motivation};

	% MDA Process - top center
	\node[processblock=mdacolor, below=of title] (mda) {
		\begin{tabular}{c}
			\textbf{MDA Process}      \\[4pt]
			\normalsize Multiple      \\
			\normalsize Displacement  \\
			\normalsize Amplification \\[3pt]
			\textcolor{mdacolor!80}{\footnotesize Nanopore Sequencing}
		\end{tabular}
	};

	% MDA outputs two types of reads
	\node[featurebox=benefitcolor, below left=of mda] (biologicalreads) {
		\begin{tabular}{c}
			\textbf{Biological}        \\[2pt]
			\textbf{True Reads}        \\[4pt]
			\footnotesize Genuine      \\
			\footnotesize genomic data \\[2pt]
			\textcolor{benefitcolor!80}{\tiny $\checkmark$ Clean}
		\end{tabular}
	};

	\node[featurebox=challengecolor, below right=of mda] (chimeraartifacts) {
		\begin{tabular}{c}
			\textbf{Chimera}             \\[2pt]
			\textbf{Artifacts}           \\[4pt]
			\footnotesize Artificial     \\
			\footnotesize MDA byproducts \\[2pt]
			\textcolor{challengecolor!80}{\tiny $\times$ Contaminant}
		\end{tabular}
	};

	% ChimeraLM filter - center position
	\node[solutionbox, below=7cm of mda] (chimeralm) {
		\begin{tabular}{c}
			\textbf{ChimeraLM}        \\[4pt]
			\textbf{AI Filter}        \\[3pt]
			\normalsize Identifies \& \\
			\normalsize Removes       \\
			\normalsize Chimera Artifacts
		\end{tabular}
	};

	% Clean downstream analysis
	\node[analysisbox=natblue, below=of chimeralm] (cleananalysis) {
		\begin{tabular}{c}
			\textbf{Clean Downstream Analysis} \\[4pt]
			\normalsize SNPs, CNVs, SVs, ...   \\[2pt]
			\textcolor{natblue!80}{\footnotesize Without artifact contamination}
		\end{tabular}
	};

	% Contaminated analysis (parallel path)
	\node[analysisbox=challengecolor, right=4cm of chimeralm] (contaminatedanalysis) {
		\begin{tabular}{c}
			\textbf{Contaminated Analysis}     \\[4pt]
			\normalsize With chimera artifacts \\[2pt]
			\textcolor{challengecolor!80}{\footnotesize False discoveries}
		\end{tabular}
	};

	% Final results
	\node[analysisbox=benefitcolor, below=of cleananalysis] (reliableresults) {
		\begin{tabular}{c}
			\textbf{Reliable Results}        \\[4pt]
			\normalsize Accurate discoveries \\[2pt]
			\textcolor{benefitcolor!80}{\footnotesize Trustworthy genomics}
		\end{tabular}
	};

	% Arrows - Main filtering path
	\draw[arrow=natgray!80] (mda.south west) to[out=225,in=90] (biologicalreads.north);
	\draw[arrow=natgray!80] (mda.south east) to[out=315,in=90] (chimeraartifacts.north);

	% Both types go to ChimeraLM
	\draw[arrow=benefitcolor] (biologicalreads.south) to[out=270,in=135] (chimeralm.north west);
	\draw[arrow=challengecolor] (chimeraartifacts.south) to[out=270,in=45] (chimeralm.north east);

	% ChimeraLM sends clean data downstream
	\draw[arrow=natpurple, line width=3pt] (chimeralm.south) -- (cleananalysis.north);

	% Clean analysis leads to reliable results  
	\draw[arrow=natblue] (cleananalysis.south) -- (reliableresults.north);

	% Alternative contaminated path (without ChimeraLM)
	\draw[arrow=challengecolor, dashed, line width=2pt] (chimeraartifacts.east) to[out=0,in=180] (contaminatedanalysis.west);

\end{tikzpicture}

\end{document}
